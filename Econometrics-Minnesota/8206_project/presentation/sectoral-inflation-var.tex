\documentclass{beamer}
\usetheme{metropolis}           % Use metropolis theme
\setbeamertemplate{itemize item}{$\diamond$}
% set the itemize subitem symbol as a triangle
\setbeamertemplate{itemize subitem}{\scriptsize$\blacktriangleright$}
% set the itemize subsubitem symbol as a circle with a dot
\setbeamertemplate{itemize subsubitem}{\scriptsize$\odot$}

\hypersetup{
    colorlinks=true,
    urlcolor=blue, % Set the URL color to dark blue
    linkcolor=red % You can also set the color for other types of links if needed
}


\title{Energy Inflation Pass Through}
\date{\today}
\author{Pranay Gundam}
\institute{University of Minnesota}
\begin{document}
  \maketitle

\begin{frame}{We typically report inflation across a bundle of goods / sectors.}
\begin{center}
	\includegraphics[width = 10.5cm]{../final_figures/aggregate_cpi_yoy_inflation_2000_present.png}
\end{center}
\end{frame}

\begin{frame}{Aggregate CPI series loses the nuance of each sector.}
\begin{center}
	\includegraphics[width = 10.5cm]{../final_figures/cpi_categories_yoy_comparison_2000_present.png}
\end{center}
\end{frame}

\begin{frame}{There does seem to be some qualitative co-movement.}
\begin{center}
	\includegraphics[width = 10.5cm]{../final_figures/cpi_categories_yoy_comparison_2018_present_twinx.png}
\end{center}
\end{frame}

%\begin{frame}{Data Description}
%\begin{itemize}
%\item BLS sectoral inflation across 5 sectors
%\begin{itemize}
%\item Seasonally adjusted

%\item Disaggregates composed as to span aggregate CPI
%\end{itemize}
%\item Expected FFR is an important component
%\end{itemize}
%\end{frame}

\begin{frame}{More recent Macro work in interaction of prices amongst sectors.}
\begin{itemize}
\item \href{https://direct.mit.edu/rest/article-abstract/98/4/770/58346/Core-Inflation-and-Trend-Inflation}{Stock and Watson(2016)} examines if measurement of trend inflation can be improved by using disaggregated sectoral inflation data.

\item \href{https://www.newyorkfed.org/medialibrary/media/research/staff_reports/sr1053.pdf?sc_lang=en}{Del Negro, di Giovanni, Dogra (2025)} studies the inflationary trends across green and dirty sectors.

\item <2> I personally feel Energy Commodities (gasoline, natural gas, etc.) is a fundamental sector that is an core input across most other sectors.
\end{itemize}
\end{frame}



\begin{frame}{This Project}
	\begin{itemize}
		\item<+-> What do I ask?
			\begin{itemize}
				\item What role do energy-commodity inflation shocks play in other industries’ inflation?
			\end{itemize}
		\item<+-> How do I do it?
			\begin{itemize}
				\item Eastimate a VAR and identify structural shocks via Cholesky Decomposition.
			\end{itemize}
		\item<+-> What do I find?
			\begin{itemize}
				\item A 1pp shock to energy-commodities inflation leads to a lagged spike in other sectoral inflation. 
			\end{itemize}
	\end{itemize}
\end{frame}

\begin{frame}{VAR Model Description}
$$X_t = \sum_{j=1}^{6}\beta_j'X_{t-j} + \epsilon_t$$
\begin{itemize}
\item $X_t$ is a $6\times 1$ vector of sectoral inflation and FFR, $\beta_j$ is a matrix of coefficients, and $\epsilon_t$ is a $6\times 1$ vector of errors
\item  6 lags, following recent suggestion by \href{https://economics.mit.edu/sites/default/files/2025-05/lp_var_primer.pdf}{Olea et al. (2025)}
\item Ordering for Cholesky Decomposition:
\end{itemize}
$$\text{Energy Commodities} \to \text{Energy Services} \to \text{Food} \to $$
$$\text{Commodities less} \to \text{Services less} \to \text{FFR}$$
\end{frame}

\begin{frame}{Primary Results}
\begin{center}
	\includegraphics[width = 9cm]{../final_figures/irf_grid_all_lags=6_yoy=true.png}
\end{center}
\end{frame}

\begin{frame}{Variance table}
\begin{center}
\begin{tabular}{ c | c }
 Sector & Variance\\
\hline 
 Energy Commodities & 445.809\\
 Energy Services & 29.9498\\
 Effective Federal Funds Rate & 12.5945\\ 
Food & 9.65913\\
 Commodities less food and energy & 4.8576\\
 Services less energy services & 1.29883
\end{tabular}
\end{center}
\end{frame}

\begin{frame}{Robustness Checks}
\textbf{What I have done:}
\begin{itemize}
\item VAR with MoM inflation.
\item Various lag lengths yield similar trend and magnitude results.
\end{itemize}
\textbf{What still could be done:}
\begin{itemize}
\item Extend the network analysis of inflation to include spatial sectors of inflation.
\item Run model on pre-pandemic data and analyze post-pandemic shock.
\item Choose lag length based on AIC.
\item Compare magnitude of response between other sectoral inflation shocks.
\end{itemize}
\end{frame}

\begin{frame}
    \centering
    \textbf{\Huge Thank You!} % Puts the text in the center, bold, and a large font size
\end{frame}
\end{document}