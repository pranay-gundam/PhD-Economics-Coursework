%--------------------
% Packages
% -------------------
\documentclass[11pt,english]{article}
\usepackage{amsfonts}
\usepackage[left=2.5cm,top=2cm,right=2.5cm,bottom=3cm,bindingoffset=0cm]{geometry}
\usepackage{amsmath, amsthm, amssymb}
\usepackage{tikz}
\usetikzlibrary{calc}
\usetikzlibrary{decorations.pathreplacing,calligraphy}
\usepackage{fancyhdr}
%\usepackage{currfile}
\usepackage{nicefrac}
\usepackage{cite}
\usepackage{graphicx}
\usepackage{caption}
\usepackage{longtable}
\usepackage{rotating}
\usepackage{lscape}
\usepackage{booktabs}
\usepackage{float}
\usepackage{placeins}
\usepackage{setspace}
\usepackage[font=itshape]{quoting}
\onehalfspacing
\usepackage{mathrsfs}
\usepackage{tcolorbox}
\usepackage{xcolor}
\usepackage{subcaption}
\usepackage{float}
\usepackage[multiple]{footmisc}
\usepackage[T1]{fontenc}
\usepackage[sc]{mathpazo}
\usepackage{listings}
\usepackage{longtable}
\definecolor{cmured}{RGB}{175,30,45}
\definecolor{macroblue}{RGB}{56,108,176}
\usepackage[format=plain,
            labelfont=bf,
            textfont=]{caption}
\usepackage[colorlinks=true,citecolor=macroblue,linkcolor=macroblue,urlcolor=macroblue]{hyperref}
\usepackage{varioref}
\usepackage{chngcntr}

\definecolor{darkgreen}{RGB}{30,175,88}
\definecolor{darkblue}{RGB}{30,118,175}
\definecolor{maroon}{rgb}{0.66,0,0}
\definecolor{darkgreen}{rgb}{0,0.69,0}

%Counters
\newtheorem{theorem}{Theorem}[section] 
\newtheorem{proposition}{Proposition}
\newtheorem{lemma}{Lemma}
\newtheorem{corollary}{Corollary}
\newtheorem{assumption}{Assumption}
\newtheorem{axiom}{Axiom}
\newtheorem{case}{Case}
\newtheorem{claim}{Claim}
\newtheorem{condition}{Condition}
\newtheorem{definition}{Definition}
\newtheorem{example}{Example}
\newtheorem{notation}{Notation}
\newtheorem{remark}{Remark}



\hypersetup{ 	
pdfsubject = {},
pdftitle = {Macro Notes},
pdfauthor = {Pranay Gundam},
linkcolor= macroblue
}


\title{\textbf{Macro Notes}}
\author{Pranay Gundam}

%-----------------------
% Begin document
%-----------------------
\begin{document}

\maketitle

\tableofcontents

\section{Introduction}

This document serves as a collection of concepts covered in graduate level macroeconomics. Specifically, the PhD Macroeconomics sequence ECON 8105 (Kehoe), ECON 8106 (Chari), ECON 8107 (Amador), and ECON 8108 (Luttmer) at the University of Minnesota. If you ever encouter these notes and have questions or spot an error, be sure to reach out to \href{mailto:pranaygundam00@gmail.com}{me here}.

\section{ECON 8105 (Kehoe)}

\subsection{General Equilibrium}

There are two main types of economic modeling strategies that we need to talk about. Arrow-Debreu (AD) representations and Sequential Markets (SM) representations. The crux of the idea is to develop a theory for the dynamic interactions of various households (and other sorts of agents) who face various resource constraints and optimization choices. We pose this question as simultaneously solving multiple maximization problems. The AD and SM representations are techniques of translating the structure of economies into a specific mathematical formulation with prices (formed based off of the relative utility preferences of agents).

In the AD representation, households are able to trade in futures contacts for some single consumption good at time zero (before the economy gets up and running) and are allowed to   

\subsection{The Canonical Basic Model}

Consider an economy that consists of one household that lives forever and wants to maximize their lifetime utility. They are endowed with some income stream that are in the units of the consumptino good.

\subsubsection{Adding Production}

Optimal growth problem

\subsubsection{Overlapping Generations}

An overlapping generations model is not necessarily a separate entity from the concept of general equilibrium. The idea is still to solve for an equilibrium but the main difference from the models we will discuss in this section versus the section above is in the lifespan of an agent. 

\subsubsection{Equivalence of Arrow-Debreu and Sequential Markets}

\subsection{Analytical Solution Methods}

\subsubsection{Algebra and Useful Places to Look}

Setting up the lagrangian.
  
Inada conditions!

Sum up the euler equations from different types of consumers to be able to plug in the market clearing conditions.

Initial fiat money is $0$ implies that bonds are $0$ for the rest of time and there is usually autarky.

\subsubsection{Negishi Method}

\subsection{Pareto Optimality Analysis}

\subsection{Deterministic Dynamic Programming}

Contractions and Blackwell's Sufficient Conditions

\section{ECON 8106 (Chari)}

\section{ECON 8107 (Amador)}

\section{ECON 8108 (Luttmer)}

\end{document}